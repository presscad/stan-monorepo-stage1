\documentclass[12pt]{article}

\title{Automatic Gradients:
\\
Extensible, Object-Oriented Reverse-Mode 
\\
Algorithmic Differentiation}
\author{Bob Carpenter and Matthew D.\ Hoffman}


\begin{document}
\maketitle
\begin{abstract}
  Given a a C++ function templated on $N$ scalar parameters,
  reverse-mode algorithmic differentiation (AD) calculates the
  function's value and gradient simultaneously in only a small
  constant multiple of the cost of evaluating the function itself.  It
  works by instantiating the template parameter to a special AD class
  and then calling the function with {\tt double} arguments.  Each
  function called with AD arguments must return an AD result that
  represents the resulting expression graph.  A backwards sweep then
  propagates partial derivatives using the chain rule from the result
  node to the input nodes in the expression graph.

  The innovation we present in this paper is to use an object-oriented
  representation of the expressions, with all internal code written in
  terms of abstract base class interfaces.  This both cleans up the
  code by localizing derivative operations to the class representing
  their expression type and opens the way for clients to extend the
  set of functions which can be algorithmically differentiated.  This
  allows client code to introduce new expression types along with
  partial derivatives, both increasing flexibility and increasing
  efficiency through partial evaluation.  We demonstrate that the
  object-oriented abstraction does not incur any computational
  overhead when the code is inlined with an optimizing compiler.
\end{abstract}

\section{Reverse-Mode Algorithmic Differentiation}

For many efficient approaches to optimization and simulation, it is
necessary to compute the gradient $\nabla f(x)$ of some scalar
function $f$ at an input $x = \langle x_1,\ldots,x_N \rangle$,
%
\[
\nabla f(x) 
= \left\langle 
\frac{\partial f(x)}{\partial x_1},
\ldots,
\frac{\partial f(x)}{\partial x_N}
\right\rangle.
\]

Algorithmic differentiation (AD) is a technique for calculating the
derivatives of a function specified as a computer program.  AD works
by building-in the partial derivative of every scalar function with
respect to each of its arguments, then using the chain rule to
propagate derivatives through the expression graph defined by the
program.  Unlike partial differences, AD is accurate up to machine
arithmetic precision.

Forward-mode algorithmic differenitation (FAD) works by propagating
derivatives forward from the inputs to the output.  FAD uses not much
more memory or time than would be required to execute the function on
floating-point inputs.  The only drawback is that FAD only calculates
a single partial derivative $\partial f(x)/\partial x_n$, requiring
$N$ calls to calculate the gradient of $f(x)$.

Reverse mode algorithmic differentiation (RAD) propagates information
in reverse from the output to the inputs.  It works by building up the
expression graph defined by the program for the function being
computed as the function is evaluated.  RAD applies dynamic
programming to propagate the partial derivatives backwards through the
expression graph all at once, allowing the partial derivatives in
the gradient $\nabla f(x)$ to be calculated in a single pass.

\subsection{Example: The Exponential Distribution}

The log of the exponential distribution $\mbox{Expon}(\theta|\beta)$
given inverse scale parameter $\beta > 0$ and outcome $\theta > 0$ 
defines a function $f(\theta,\beta)$ defined by
%
\[
f(\theta,\beta) 
= \log \mbox{Expon}(\theta|\beta)
= \log(\beta)  - ( \beta \times \theta ).
\]
%

\[
\begin{array}{lllll}
\mbox{\it Node} & \mbox{\it Function} & \mbox{\it Inputs} & \mbox{\it
  Partials} & \mbox{\it Expression}
\\
x_0 &  \mbox{n/a} & \mbox{n/a} & \mbox{n/a} & \theta
\\
x_1 &  \mbox{n/a} & \mbox{n/a} & \mbox{n/a} & \alpha
\\
x_2 &  \log {} & x_0 & 1/x_0 & \log(\beta)
\\
x_3 &  {} \times {} & x_1; \ x_0 & 
\frac{\partial (x_1 \times x_0)}{\partial x_1} = x_0; \
\frac{\partial (x_1 \times x_0)}{\partial x_0} = x_1
& \beta \times \theta
\\[6pt]
x_4 &  {} - {} & x_2; \ x_3 & 
\frac{\partial (x_2 -x_3)}{\partial x_2} = 1; \ 
\frac{\partial (x_2 - x_3)}{\partial x_3} = -1
& \log(\beta) - (\beta \times \theta)
\end{array}
\]

\end{document}
