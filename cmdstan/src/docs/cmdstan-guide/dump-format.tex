\chapter{Dump Data Format}\label{dump.chapter}

\noindent 
For representing structured data in files, \CmdStan uses the dump format
introduced in \SPLUS and used in \R and \JAGS (and in \BUGS, but with
a different ordering).   A dump file is structured as a sequence of
variable definitions.  Each variable is defined in terms of its
dimensionality and its values.   There are three kinds of variable
declarations, one for scalars, one for sequences, and one for general
arrays.

\section{Creating Dump Files}

Dump files can be created from R using RStan.  The function is
\code{stan\_rdump} in package \code{rstan}.

Using R's native \code{dump()} function can produce dump files which
Stan cannot read in.  The underlying cause is that R supports
complicated data structures, some of which are not used in \CmdStan.
For example, R's \code{dump()} can write a numerical vector with names
for each element.

\section{Scalar Variables}

A simple scalar value can be thought of as having an empty list of
dimensions.  Its declaration in the dump format follows the \SPLUS
assignment syntax.  For example, the following would constitute a
valid dump file defining a single scalar variable \code{y} with value
17.2.
%
\begin{quote}
\begin{Verbatim}
y <- 
17.2
\end{Verbatim}
\end{quote}
%
A scalar value is just a zero-dimensional array value.


\section{Sequence Variables}\label{sequence-variables.section}

One-dimensional arrays may be specified directly using the \SPLUS
sequence notation.  The following example defines an integer-value and
a real-valued sequence.
%
\begin{quote}
\begin{Verbatim}
n <- c(1,2,3)
y <- c(2.0,3.0,9.7)
\end{Verbatim}
\end{quote}
%
Arrays are provided without a declaration of dimensionality because
the reader just counts the number of entries to determine the size of
the array.

Sequence variables may alternatively be represented with \R's
colon-based notation.  For instance, the first example above could
equivalently be written as
%
\begin{quote}
\begin{Verbatim} 
n <- 1:3
\end{Verbatim}
\end{quote}
% 
The sequence denoted by \code{1:3} is of length 3, running from 1 to 3
inclusive.  The colon notation allows sequences going from high to
low, as in the first of the following examples, which is equivalent to
the second.
%
\begin{quote}
\begin{Verbatim}
n <- 2:-2
n <- c(2,1,0,-1,-2)
\end{Verbatim}
\end{quote}
%

As a special case, a sequence of zeros can also be
represented in the dump format by \code{integer(x)} and
\code{double(x)}, for type int and double, respectively.
Here \code{x} is a non-negative integer to specify the
length. If \code{x} is \code{0}, it can be ommitted. The
following are some examples.
%
\begin{quote}
\begin{Verbatim}
x1 <- integer()
x2 <- integer(0)
x3 <- integer(2)
y1 <- double()
y2 <- double(0)
y3 <- double(2)
\end{Verbatim}
\end{quote}
%


\section{Array Variables}\label{array-variables.section}

For more than one dimension, the dump format uses a dimensionality
specification.  For example,
%
\begin{quote}
\begin{verbatim}
y <- structure(c(1,2,3,4,5,6), .Dim = c(2,3))
\end{verbatim}
\end{quote}
%
This defines a $2 \times 3$ array.  Data is stored in column-major
order, meaning the values for \code{y} will be as follows.
%
\begin{quote}
\begin{Verbatim}
y[1,1] = 1     y[1,2] = 3     y[1,3] = 5    
y[2,1] = 2     y[2,2] = 4     y[2,3] = 6
\end{Verbatim}
\end{quote}
%
The \code{structure} keyword just wraps a sequence of values and a
dimensionality declaration, which is itself just a sequence of
non-negative integer values.  The product of the dimensions must equal
the length of the array.

If the values happen to form a contiguous sequence of integers,
they may be written with colon notation.  Thus the example above is
equivalent to the following.
%
\begin{quote}
\begin{verbatim}
y <- structure(1:6, .Dim = c(2,3))
\end{verbatim}
\end{quote}
%
The same applies to the specification of dimensions, though it is
perhaps less likely to be used. In the above example,
c(2,3) could be written as \code{2:3}.

Arrays of more than two dimensions are written in a last-index major form.
For example, 
%
\begin{quote}
\begin{verbatim}
z <- structure(1:24, .Dim = c(2,3,4))
\end{verbatim}
\end{quote}
%
produces a three-dimensional \code{int} (assignable to \code{real})
array \code{z} with values
%
\begin{quote}
\begin{verbatim}
z[1,1,1] =  1   z[1,2,1] =  3   z[1,3,1] =  5
z[2,1,1] =  2   z[2,2,1] =  4   z[2,3,1] =  6

z[1,1,2] =  7   z[1,2,2] =  9   z[1,3,2] = 11
z[2,1,2] =  8   z[2,2,2] = 10   z[2,3,2] = 12

z[1,1,3] = 13   z[1,2,3] = 15   z[1,3,3] = 17
z[2,1,3] = 14   z[2,2,3] = 16   z[2,3,3] = 18

z[1,1,4] = 19   z[1,2,4] = 21   z[1,3,4] = 23
z[2,1,4] = 20   z[2,2,4] = 22   z[2,3,4] = 24
\end{verbatim}
\end{quote}

The sequence of values inside \code{structure} can also be
\code{integer(x)} or \code{double(x)}. In particular, if one
or more dimensions is zero, \code{integer()} can be put inside
\code{structure}.  For instance, the following example is supported
by the dump format.

\begin{quote}
\begin{verbatim}
y <- structure(integer(), .Dim = c(2, 0))
\end{verbatim}
\end{quote}


\section{Matrix- and Vector-Valued Variables}

The dump format for matrices and vectors, including arrays of matrices
and vectors, is the same as that for arrays of the same shape.

\subsection{Vector Dump Format}

The following three declarations have the same dump format for their
data.
%
\begin{quote}
\begin{Verbatim}
real a[K];
vector[K] b;
row_vector[K] c;
\end{Verbatim}
\end{quote}

\subsection{Matrix Dump Format}

The following declarations have the same dump format.
%
\begin{quote}
\begin{Verbatim}
real a[M,N];
matrix[M,N] b;
\end{Verbatim}
\end{quote}

\subsection{Arrays of Vectors and Matrices}

The key to undertanding arrays is that the array indexing comes before
any of the container indexing.  That is, an array of vectors is just
that --- provide an index and get a vector.  See the chapter on array and matrix types in the user's guide section of the languag emanual for more information.

For the dump data format, the following declarations have the same
arrangement.
%
\begin{quote}
\begin{Verbatim}
real a[M,N];
matrix[M,N] b;
vector[N] c[M];
row_vector[N] d[M];
\end{Verbatim}
\end{quote}
%
Similarly, the following also have the same dump format.
%
\begin{quote}
\begin{Verbatim}
real a[P,M,N];
matrix[M,N] b[P];
vector[N] c[P,M];
row_vector[N] d[P,M];
\end{Verbatim}
\end{quote}

\section{Integer- and Real-Valued Variables}

There is no declaration in a dump file that distinguishes integer
versus continuous values.  If a value in a dump file's definition of a
variable contains a decimal point (e.g., \code{132.3}) or uses
scientific notation (e.g., \code{1.323e2}), Stan assumes that the
values are real.

For a single value, if there is no decimal point, it may be assigned
to an \code{int} or \code{real} variable in Stan.  An array value may
only be assigned to an \code{int} array if there is no decimal point
or scientific notation in any of the values.  This convention is
compatible with the way \R writes data.

The following dump file declares an integer value for \code{y}.
%
\begin{quote}
\begin{Verbatim} 
y <- 
2
\end{Verbatim}
\end{quote}
% 
This definition can be used for a Stan variable \code{y} declared as
\code{real} or as \code{int}.  Assigning an integer value to a real
variable automatically promotes the integer value to a real value.

Integer values may optionally be followed by \code{L} or \code{l},
denoting long integer values.  The following example, where the type is
explicit, is equivalent to the above.
%
\begin{quote}
\begin{Verbatim} 
y <- 
2L
\end{Verbatim}
\end{quote}

The following dump file provides a real value for \code{y}.
%
\begin{quote}
\begin{Verbatim}
y <-
2.0
\end{Verbatim}
\end{quote}
%
Even though this is a round value, the occurrence of the decimal
point in the value, \code{2.0}, causes Stan to infer that \code{y} is
real valued.  This dump file may only be used for variables \code{y}
declared as real in Stan.

\subsection{Scientific Notation}

Numbers written in scientific notation may only be used for real
values in Stan.  R will write out the integer one million as
\code{1e+06}.  




\subsection{Infinite and Not-a-Number Values}

Stan's reader supports infinite and not-a-number values for scalar
quantities (see the section of the reference manual section of the
language manaul for more information on Stan's numerical data types).
Both infinite and not-a-number values are supported by Stan's
dump-format readers.
%
\begin{center}
\begin{tabular}{r||c|c}
{\it Value} & {\it Preferred Form} & {\it Alternative Forms} \\ \hline \hline
positive infinity & \code{Inf} & \code{Infinity},
\code{infinity}
\\
negative infinity & \code{-Inf} & \code{-Infinity},
\code{-infinity}
\\
not a number & \code{NaN} & 
\end{tabular}
\end{center}
%
These strings are not case sensitive, so \code{inf} may also be used
for positive infinity, or \code{NAN} for not-a-number.

\section{Quoted Variable Names}

In order to support \JAGS data files, variables may be double quoted.
For instance, the following definition is legal in a dump file.
%
\begin{quote}
\begin{Verbatim}
"y" <-
c(1,2,3)
\end{Verbatim}
\end{quote}

\section{Line Breaks}

The line breaks in a dump file are required to be consistent with
the way \R reads in data.  Both of the following declarations are
legal.
%
\begin{quote}
\begin{Verbatim}
y <- 2
y <-
3
\end{Verbatim}
\end{quote}
%
Also following \R, breaking before the assignment arrow are not
allowed, so the following is invalid.
%
\begin{quote}
\begin{Verbatim}
y
<- 2  # Syntax Error
\end{Verbatim}
\end{quote}

Lines may also be broken in the middle of sequences declared
using the \code{c(...)} notation., as well as between the comma
following a sequence definition and the dimensionality declaration.
For example, the following declaration of a $2 \times 2 \times 3$
array is valid.
%
\begin{quote}
\begin{Verbatim}
y <-
structure(c(1,2,3,
4,5,6,7,8,9,10,11,
12), .Dim = c(2,2,
3))
\end{Verbatim}
\end{quote}
%
Because there are no decimal points in the values, the resulting dump
file may be used for three-dimensional array variables declared as
\code{int} or \code{real}.

\section{BNF Grammar for Dump Data}

A more precise definition of the dump data format is provided
by the following (mildly templated) Backus-Naur form grammar.

{\small 
\begin{verbatim}
 definitions ::= definition+

 definition ::= name ("<-" | '=') value optional_semicolon

 name ::= char* 
        | ''' char* ''' 
        | '"' char* '"'

 value ::= value<int> | value<double>

 value<T> ::= T 
            | seq<T>
            | zero_array<T>
            | 'structure' '(' seq<T> ',' ".Dim" '=' seq<int> ')'
            | 'structure' '(' zero_array<T> ',' ".Dim" '=' seq<int> ')'

 seq<int> ::= int ':' int
            | cseq<int>

 zero_array<int> ::= "integer" '(' <non-negative int>? ')'

 zero_array<real> ::= "double" '(' <non-negative int>? ')'

 seq<real> ::= cseq<real>

 cseq<T> ::= 'c' '(' vseq<T> ')'

 vseq<T> ::= T
           | T ',' vseq<T>
\end{verbatim}
}
\noindent
The template parameters \code{T} will be set to either \code{int} or
\code{real}.  Because Stan allows promotion of integer values to real
values, an integer sequence specification in the dump data format may
be assigned to either an integer- or real-based variable in Stan.


