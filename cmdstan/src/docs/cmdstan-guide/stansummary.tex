\chapter{{\tt\bfseries print}: Output Analysis (deprecated)}\label{print.chapter}

\noindent
\code{print} is deprecated, but is still available until CmdStan v3.0.
See the next chapter for usage (replace \code{stansummary} with
\code{print}).

\chapter{{\tt\bfseries stansummary}: Output Analysis}\label{stansummary.chapter}

\noindent
\CmdStan is distributed with a posterior analysis utility that is able to read in
the output of one or more Markov chains and summarize the posterior
fits.  This operation mimics the \code{print(fit)} command in RStan,
which itself was modeled on the print functions from R2WinBUGS and
R2jags.

\section{Building the stansummary Command}

\CmdStan's \code{stansummary} command is built along with \code{stanc} into
the \code{bin} directory. It can be compiled directly using the
makefile as follows.
%
\begin{quote}
\begin{Verbatim}[fontshape=sl]
> cd <cmdstan-home>
> make bin/stansummary
\end{Verbatim}
\end{quote}
%

\section{Running the stansummary Command}

The \code{stansummary} command is executed on one or more \code{output.csv}
files. These files may be provided as command-line arguments
separated by spaces.  That means that wildcards may be used, as they
will be replaced by space-separated file names by the operating
system's command-line interpreter.

Suppose there are three samples files in a directory generated by
fitting a negative binomial model to a small data set.
%
\begin{quote}
\begin{Verbatim}[fontshape=sl]
> ls output*.csv
\end{Verbatim}
%
\begin{Verbatim}
output1.csv	output2.csv	output3.csv
\end{Verbatim}
%
\begin{Verbatim}[fontshape=sl]
> bin/stansummary output*.csv
\end{Verbatim}
\end{quote}
%
The result of \code{bin/stansummary} is displayed in
\reffigure{bin-stansummary-eg}.%
%
\footnote{RStan's and PyStan's output analysis stansummary may be different
  than that in the command-line version of Stan.}
%
\begin{figure}
\begin{Verbatim}[fontsize=\footnotesize]
   Inference for Stan model: negative_binomial_model
   1 chains: each with iter=(1000); warmup=(0); thin=(1); 1000 iterations saved.

   Warmup took (0.054) seconds, 0.054 seconds total
   Sampling took (0.059) seconds, 0.059 seconds total

                   Mean     MCSE   StdDev    5%   50%   95%  N_Eff  N_Eff/s    R_hat
   lp__             -14  7.0e-02  1.1e+00   -17   -14   -13    226     3022  1.0e+00
   accept_stat__   0.94  3.1e-03  9.7e-02  0.75  0.98   1.0   1000    13388  1.0e+00
   stepsize__      0.16  5.1e-16  3.6e-16  0.16  0.16  0.16   0.50      6.7  1.0e+00
   treedepth__      2.9  4.1e-02  1.2e+00   1.0   3.0   5.0    829    11104  1.0e+00
   n_leapfrog__     8.0  2.1e-01  6.3e+00   1.0   7.0    19    870    11648  1.0e+00
   divergent__     0.00  0.0e+00  0.0e+00  0.00  0.00  0.00   1000    13388      nan
   energy__          15  8.7e-02  1.5e+00    14    15    18    282     3775  1.0e+00
   alpha             16  1.9e+00  2.0e+01   1.9   9.7    50    114     1524  1.0e+00
   beta             9.9  1.1e+00  1.2e+01   1.1   6.1    31    124     1664  1.0e+00

   Samples were drawn using hmc with nuts.
   For each parameter, N_Eff is a crude measure of effective sample size,
   and R_hat is the potential scale reduction factor on split chains (at 
   convergence, R_hat=1).
\end{Verbatim}
\vspace*{-6pt}
\caption{\small\it Example output from \code{bin/stansummary}.  The model
  parameters are \code{alpha} and \code{beta}.  The values for each
  quantity are the posterior means, standard deviations, and
  quantiles, along with Monte-Carlo standard error, effective sample
  size estimates (per second), and convergence diagnostic statistic.
  These values are all estimated from samples. In addition to the
  parameters, \code{bin/stansummary} also outputs \code{lp\_\_}, 
  the total log probability density (up to an additive constant) at each sample,
  as well as NUTS-specific values that can be helpful in diagnostics. 
  The quantity \code{accept\_stat\_\_} is the average Metropolis acceptance 
  probability over each simulated Hamiltonian trajectory and \code{stepsize\_\_} 
  is the integrator step size used in each simulation.  \code{treedepth\_\_} is the 
  depth of tree used by NUTS while \code{n\_leapfrog\_\_} is the number of leapfrog 
  steps taken during the Hamiltonian simulation; \code{treedepth\_\_} should always
  be the binary log of \code{n\_leapfrog\_\_}.  \code{divergent\_\_} indicates
  whether or not the simulated Hamiltonian trajectory became unstable and
  diverged.  Finally, \code{energy\_\_} is value of the Hamiltonian (up to an additive
  constant) at each sample, also known as the energy.
  }
\label{bin-stansummary-eg.figure}
\end{figure}
%\end{quote}
%
The posterior is skewed to the high side, resulting in posterior means
($\alpha=17$ and $\beta=10$) that are a long way away from the posterior
medians ($\alpha=9.5$ and $\beta=6.2$);  the posterior median is the
value listed under \code{50\%}, which is the 50th percentile of the
posterior values.

For Windows, the forward slash in paths need to be converted to backslashes.


\subsection{Output of stansummary Command}

\subsubsection{\code{divergent}}

\CmdStan uses a symplectic integrator to approximate the exact
solution of the Hamiltonian dynamics, and when the step size is too
large relative to the curvature of the log posterior this
approximation becomes unstable and the trajectories can diverge and
threaten the validity of the sampler; \code{divergent} indicates whether
or not a given trajectory diverged.  If there are any divergences then 
the samples may be biased -- common solutions are decreasing the
step size (often by increasing the target average acceptance probability)
or reparameterizing the model.

\subsubsection{\code{energy}}

The energy, \code{energy}, is used to diagnose the accuracy of any Hamiltonian
Monte Carlo sampler.  If the standard deviation of \code{energy} is much larger
than $\sqrt{D / 2}$, where $D$ is the number of \emph{unconstrained}
parameters, then the sampler is unlikely to be able to explore the
posterior adequately.  This is usually due to heavy-tailed posteriors and
can sometime be remedied by reparameterizing the model.

\section{Command-line Options}

In addition to the filenames, \code{stansummary} includes three flags to
customize the output.

\begin{description}
\longcmd{help} {stansummary usage information} {No help output by default}
%
\cmdarg{sig\_figs}{int}
{Sets the number of significant figures displayed in the output}
{Valid values: 0 \textless sig\_figs}
{default = \code{2}}
%
\cmdarg{autocorr}{int}
{Calculates and then displays the autocorrelation of the specified chain}
{Valid values: Any integer matching a chain index}
{No autocorrelation output by default}
%
\cmdarg{csv\_file}{string}
       {Writes output as a csv file with comments written as \#}
       {Valid values: Any valid filename}
       {Appends output to the file if it exists}
\end{description}
